%----------------------------------------------------------------------------------------
%	PACKAGES AND OTHER DOCUMENT CONFIGURATIONS
%----------------------------------------------------------------------------------------

\documentclass[
10pt, % Main document font size
a4paper, % Paper type, use 'letterpaper' for US Letter paper
oneside, % One page layout (no page indentation)
%twoside, % Two page layout (page indentation for binding and different headers)
headinclude,footinclude, % Extra spacing for the header and footer
BCOR5mm, % Binding correction
]{scrartcl}

%----------------------------------------------------------------------------------------
%	REQUIRED PACKAGES
%----------------------------------------------------------------------------------------

\usepackage[
nochapters, % Turn off chapters since this is an article        
beramono, % Use the Bera Mono font for monospaced text (\texttt)
eulermath,% Use the Euler font for mathematics
pdfspacing, % Makes use of pdftex’ letter spacing capabilities via the microtype package
dottedtoc % Dotted lines leading to the page numbers in the table of contents
]{classicthesis} % The layout is based on the Classic Thesis style

\usepackage{arsclassica} % Modifies the Classic Thesis package
\usepackage[T1]{fontenc} % Use 8-bit encoding that has 256 glyphs
\usepackage[italian]{babel}
\usepackage[utf8]{inputenc} % Required for including letters with accents
\usepackage{graphicx} % Required for including images
\graphicspath{{img/}} % Set the default folder for images
\usepackage{enumitem} % Required for manipulating the whitespace between and within lists
\usepackage{lipsum} % Used for inserting dummy 'Lorem ipsum' text into the template
\usepackage{subfig} % Required for creating figures with multiple parts (subfigures)
\usepackage{amsmath,amssymb,amsthm} % For including math equations, theorems, symbols, etc
\usepackage{varioref} % More descriptive referencing
\usepackage[dvipsnames]{xcolor} % Colors and other amenities
\usepackage{listings} % Code
\usepackage{tikz}
\usepackage{pgfplots}
\usepackage{bm}
%----------------------------------------------------------------------------------------
%	CODE STYLE
%---------------------------------------------------------------------------------------

\lstset{language=Java,
  frame=single,
  basicstyle=\ttfamily,
  keywordstyle=\color{BrickRed},
  commentstyle=\color{gray!80!white},
  morecomment=[l]{!\ },
  breaklines=true
}


%----------------------------------------------------------------------------------------
%	THEOREM STYLES
%---------------------------------------------------------------------------------------

\theoremstyle{definition} % Define theorem styles here based on the definition style (used for definitions and examples)
\newtheorem{definition}{Definition}

\theoremstyle{plain} % Define theorem styles here based on the plain style (used for theorems, lemmas, propositions)
\newtheorem{theorem}{Theorem}

\theoremstyle{remark} % Define theorem styles here based on the remark style (used for remarks and notes)

%----------------------------------------------------------------------------------------
%	HYPERLINKS
%---------------------------------------------------------------------------------------

\hypersetup{
%draft, % Uncomment to remove all links (useful for printing in black and white)
colorlinks=true, breaklinks=true, bookmarks=true,bookmarksnumbered,
urlcolor=webbrown, linkcolor=RoyalBlue, citecolor=webgreen, % Link colors
pdftitle={}, % PDF title
pdfauthor={\textcopyright}, % PDF Author
pdfsubject={}, % PDF Subject
pdfkeywords={Casulli, Zanolli, Richards, 1d, 2d, Java, OMS3}, % PDF Keywords
pdfcreator={pdfLaTeX}, % PDF Creator
pdfproducer={LaTeX with hyperref and ClassicThesis} % PDF producer
} % Include the structure.tex file which specified the document structure and layout

\hyphenation{Java hy-phen-ation} % Specify custom hyphenation points in words with dashes where you would like hyphenation to occur, or alternatively, don't put any dashes in a word to stop hyphenation altogether

\pgfmathdeclarefunction{gauss}{2}{%
  \pgfmathparse{1/(#2*sqrt(2*pi))*exp(-((x-#1)^2)/(2*#2^2))}%
}
\makeatletter
    \pgfmathdeclarefunction{erf}{1}{%
        \begingroup
            \pgfmathparse{#1 > 0 ? 1 : -1}%
            \edef\sign{\pgfmathresult}%
            \pgfmathparse{abs(#1)}%
            \edef\x{\pgfmathresult}%
            \pgfmathparse{1/(1+0.3275911*\x)}%
            \edef\t{\pgfmathresult}%
            \pgfmathparse{%
                1 - (((((1.061405429*\t -1.453152027)*\t) + 1.421413741)*\t 
                -0.284496736)*\t + 0.254829592)*\t*exp(-(\x*\x))}%
            \edef\y{\pgfmathresult}%
            \pgfmathparse{(\sign)*\y}%
            \pgfmath@smuggleone\pgfmathresult%
        \endgroup
    }
\makeatother
\pgfmathdeclarefunction{cdf}{2}{%
  \pgfmathparse{0.5*(1+(erf((x-#1)/(#2*sqrt(2)))))}%
}

%----------------------------------------------------------------------------------------
%	TITLE AND AUTHOR(S)
%----------------------------------------------------------------------------------------

\title{\normalfont\spacedallcaps{Documentazione codice Richards 1-2 D}} % The article title

\author{\spacedlowsmallcaps{Riccardo Rigon*, Francesco Serafin* \& Aaron Iemma*}} % The article author(s) - author affiliations need to be specified in the AUTHOR AFFILIATIONS block

\date{\today} % An optional date to appear under the author(s)

%----------------------------------------------------------------------------------------

\begin{document}

%----------------------------------------------------------------------------------------
%	HEADERS
%----------------------------------------------------------------------------------------

\renewcommand{\sectionmark}[1]{\markright{\spacedlowsmallcaps{#1}}} % The header for all pages (oneside) or for even pages (twoside)
\newcommand{\subroutines}{\noindent\textbf{\spacedlowsmallcaps{CHIAMATE AD ALTRE SUBROUTINE RILEVANTI}}}
%\renewcommand{\subsectionmark}[1]{\markright{\thesubsection~#1}} % Uncomment when using the twoside option - this modifies the header on odd pages
\lehead{\mbox{\llap{\small\thepage\kern1em\color{halfgray} \vline}\color{halfgray}\hspace{0.5em}\rightmark\hfil}} % The header style
\makeatletter
\newcommand*{\inlineequation}[2][]{%
  \begingroup
    % Put \refstepcounter at the beginning, because
    % package `hyperref' sets the anchor here.
    \refstepcounter{equation}%
    \ifx\\#1\\%
    \else
      \label{#1}%
    \fi
    % prevent line breaks inside equation
    \relpenalty=10000 %
    \binoppenalty=10000 %
    \ensuremath{%
      % \displaystyle % larger fractions, ...
      #2%
    }%
    ~\@eqnnum
  \endgroup
}
\makeatother


\titleformat{\subsection}{\relax}{\textsc{\MakeTextLowercase{\thesubsection}}}{1em}{\normalsize\bfseries}

\pagestyle{scrheadings} % Enable the headers specified in this block

%----------------------------------------------------------------------------------------
%	TABLE OF CONTENTS & LISTS OF FIGURES AND TABLES
%----------------------------------------------------------------------------------------

\maketitle % Print the title/author/date block
\setcounter{tocdepth}{2} % Set the depth of the table of contents to show sections and subsections only
\tableofcontents % Print the table of contents

%----------------------------------------------------------------------------------------
%	ABSTRACT
%----------------------------------------------------------------------------------------

\section*{Abstract} % This section will not appear in the table of contents due to the star (\section*)

Si riporta di seguito documentazione del codice \texttt{Java} prodotto dall'implementazione
di \cite{Casulli2010} all'interno del framework OMS3. 

%----------------------------------------------------------------------------------------
%	AUTHOR AFFILIATIONS
%----------------------------------------------------------------------------------------

{\let\thefootnote\relax\footnotetext{* \textit{Dipartimento di Ingegneria Civile ed Ambientale, Università degli Studi di Trento, Trento, Italy}}}
\section{Note}
\begin{itemize}\itemsep0pt
	\item dato che i volumi di controllo (vedi \cite{Casulli2010} pp. 2257) devono avere facce di contorno normali alla linea che connette i centri di ogni volume adiacente (quantomeno per i centri che sono in "vista diretta" l'uno rispetto all'altro, \emph{e.g.}, la cui linea di connessione non interseca nessun altro volume tranne quelli a cui i punti appartengono), gli stessi volumi di controllo devono essere ottenuti tramite una scomposizione in un diagramma di Voronoi (in metrica euclidea) del dominio!
	\item La conducibilità idraulica/permeablità, dato che \\\inlineequation[maxK]{\mathcal{K}_{ j}^{ n}=\mathcal{A}_{ j}max\left[K_{ i}(\psi_{ i}^{ n}),K_{\wp({ i},{ j})}(\psi_{\wp({ i},{ j})}^{ n})\right]}, "\emph{diffonde}" dalle celle adiacenti in cui è maggiore alle celle adiacenti in cui è minore, tendendo ad assumere gli stessi valori su tutto il campo.\\
	Da \ref{maxK}, per una \emph{mesh} molto semplificata ove i numeri nelle celle rappresentano dei valori di $K$, ho ad esempio: \\
	\\
		\begin{figure}[!h]
		\centering
		\subfloat[Iter. 1]{
			\begin{tikzpicture}
				\draw[step=0.5cm,color=gray] (-1,-1) grid (0.5,0.5);
					\node at (-0.75,+0.25) {1};
					\node at (-0.25,+0.25) {4};
					\node at (+0.25,+0.25) {2};
					\node at (-0.75,-0.25) {2};
					\node at (-0.25,-0.25) {1};
					\node at (+0.25,-0.25) {3};
					\node at (-0.75,-0.75) {1};
					\node at (-0.25,-0.75) {3};
					\node at (+0.25,-0.75) {2};
				\end{tikzpicture}
		}
		\hspace{1cm}
		\subfloat[Iter. 2]{		
			\begin{tikzpicture}
			\draw[step=0.5cm,color=gray] (-1,-1) grid (0.5,0.5);
				\node at (-0.75,+0.25) {4};
				\node at (-0.25,+0.25) {4};
				\node at (+0.25,+0.25) {4};
				\node at (-0.75,-0.25) {2};
				\node at (-0.25,-0.25) {4};
				\node at (+0.25,-0.25) {3};
				\node at (-0.75,-0.75) {3};
				\node at (-0.25,-0.75) {3};
				\node at (+0.25,-0.75) {3};
			\end{tikzpicture}
		}
		\hspace{1cm}
		\subfloat[Iter. 3]{		
			\begin{tikzpicture}
			\draw[step=0.5cm,color=gray] (-1,-1) grid (0.5,0.5);
				\node at (-0.75,+0.25) {4};
				\node at (-0.25,+0.25) {4};
				\node at (+0.25,+0.25) {4};
				\node at (-0.75,-0.25) {4};
				\node at (-0.25,-0.25) {4};
				\node at (+0.25,-0.25) {4};
				\node at (-0.75,-0.75) {3};
				\node at (-0.25,-0.75) {4};
				\node at (+0.25,-0.75) {3};
			\end{tikzpicture}
		}
		\hspace{1cm}
		\subfloat[Iter. 4]{		
			\begin{tikzpicture}
			\draw[step=0.5cm,color=gray] (-1,-1) grid (0.5,0.5);
				\node at (-0.75,+0.25) {4};
				\node at (-0.25,+0.25) {4};
				\node at (+0.25,+0.25) {4};
				\node at (-0.75,-0.25) {4};
				\node at (-0.25,-0.25) {4};
				\node at (+0.25,-0.25) {4};
				\node at (-0.75,-0.75) {4};
				\node at (-0.25,-0.75) {4};
				\node at (+0.25,-0.75) {4};
			\end{tikzpicture}	
		}
		\end{figure}						
\end{itemize}

\section{SUBROUTINES Reologia - Rheology}

	%------------------------------------------------------------------------------------
	% SUBROUTINE KAPPA
	%------------------------------------------------------------------------------------

	\subsection{kappa}
		La funzione esplicita la relazione costitutiva di Van Genuchten, chiamando la funzione
		\texttt{Thetaf} \ref{sub:thetaf} per il calcolo del contenuto d'acqua $\theta$ in dipendenza dall'
		altezza piezometrica $\psi$ (da notare quindi che $\theta(\psi)$).
		A seconda che le condizioni siano di saturazione $\psi>0$ o meno $\psi\leq0$, la funzione procede
		al calcolo della conducibilità idraulica $K$.
		Come praticamente tutte le funzioni che concernono la reologia, questa opera diversamente a seconda
		che ci si trovi
		\begin{description}\itemsep0pt
			\item[in saturazione], dove la K assume il valore della conducibilità idraulica a saturazione $K_{s}$;
			\item[in zona insatura], dove la K, dipendente dall'altezza piezometrica $\psi$, viene computata 
			tramite la funzione:
			\begin{equation}
				K(\psi) = K_{s}*\left[\frac{\theta-\theta_{r}}{\theta_{s}-\theta_{r}}\right]^{\frac{1}{2}}*\left\{1-\left[1-\left(\frac{\theta-\theta_{r}}{\theta_{s}-\theta_{r}}\right)^{\frac{1}{m}}\right]^{m}\right\}^{2}
			\label{eq:kappainsat}
			\end{equation}
		\end{description}

		Per brevità, la saturazione effettiva nel codice viene chiamata $Se = \frac{\theta-\theta_{r}}{\theta_{s}-\theta_{r}}$,
		rendendo la formula di cui sopra più compatta e leggibile:
			\begin{equation}
				K(\psi) = K_{s}*Se^{\frac{1}{2}}*\left[1-\left(1-Se^{\frac{1}{m}}\right)^{m}\right]^{2}
			\label{eq:kappainsatse}
			\end{equation} 
		\dots con m parametro del terreno da calibrare.

		%\lstinputlisting[language=Fortran, firstline=5, lastline=19]{../BASE_CODE/Rheology.f90}
			\subroutines
			%\lstinputlisting[language=Fortran, firstline=11, lastline=11]{../BASE_CODE/Rheology.f90} La \emph{subroutine}
		 	calcola il $\theta$ (\texttt{th} nel codice) relativo ai parametri forniti in ingresso.		
	%------------------------------------------------------------------------------------
	% SUBROUTINE THETAF
	%------------------------------------------------------------------------------------

	\subsection{Thetaf}
	\label{sub:thetaf}
		La funzione calcola il contenuto d'acqua in base alla data altezza piezometrica tramite la parametrizzazione di van Genuchten \cite{Genuchten1980}, provvedento:
		\begin{description}\itemsep0pt
			\item[in saturazione], ad assegnare a $\theta$ il $\theta_{s}$ a saturazione;
			\item[nella zona insatura], a calcolare $\theta$ con l'equazione:
			\begin{equation}
				\theta(\psi) = \theta_{r} + \frac{\theta_{s}-\theta_{r}}{1-|\alpha\psi|^{n}}^{m}
				\label{eq:thetainsat}
			\end{equation}
		\end{description}
		\dots con il seguente significato dei simboli:
		\begin{itemize}\itemsep0pt
			\item $\theta_{r}$: contenuto d'acqua residuo (si veda la teoria sulle \texttt{SWRC}, \emph{Soil Water Retention Curves})
			\item $\theta_{s}$: contenuto d'acqua a saturazione
			\item $\alpha$ ($\geq0$), m e n parametri del terreno da calibrare, con:
			\begin{description}\itemsep0pt
				\item[$\bm{\alpha}$]: $>0$, relazionato all'inverso della pressione di suzione d'entrata dell'aria; 
				\item[n]: una misura della distribuzione delle grandezze dei pori;
				\item[m]: {$1-\frac{1}{n}$}
			\end{description}
		\end{itemize}			
		%\lstinputlisting[language=Fortran, firstline=22, lastline=32]{../BASE_CODE/Rheology.f90}


	%------------------------------------------------------------------------------------
	% SUBROUTINE DTHETA
	%------------------------------------------------------------------------------------
	\subsection{dTheta}
		La funzione calcola la capacità specifica del volume di controllo, l'integrale della quale fornisce il 
		contenuto d'acqua dello stesso volume.

		In formule:
		\begin{equation}
			c(\psi) = \frac{\partial \theta}{\partial \psi} ; \theta(\psi) = \theta_{r} + \int_{-\infty}^{\psi}c(\psi)d\psi
			\label{eq:speccap}
		\end{equation}
		La capacità è calcolata tramite la seguente equazione parametrica:
		\begin{description}\itemsep0pt
			\item[in saturazione] ($\psi > 0$)\newline $c(\psi) = 0$ - ovvero: "il volume dei vuoti residuo è nullo, quindi non è possibile
			saturarlo ulteriormente!"
			\item[nella zona insatura]
			\begin{equation}
				c(\psi) = \alpha\cdot n \cdot m \cdot \frac{\theta_{s} - \theta_{r}}{(1+|\alpha\psi|^{n})^{m+1}}|\alpha\psi|^{n-1}
			\end{equation}
		\end{description}
		%\lstinputlisting[language=Fortran, firstline=59, lastline=69]{../BASE_CODE/Rheology.f90}

	%------------------------------------------------------------------------------------
	% SUBROUTINE DTHETA1
	%------------------------------------------------------------------------------------
	\subsection{dTheta1}
	\label{subs:theta1}
		La funzione calcola la derivata del primo elemento della decomposizione di Jordan del contenuto d'acqua , quindi ottenendo la 
		capacità specifica del primo elemento della decomposizione $p(\psi)=\frac{\partial\theta_{1}}{\partial\psi}$.

		Nel dettaglio ho, dato che ho assunto (\cite{Casulli2010}, pp. 2258) che la capacità specifica abbia delle proprietà per cui
		sia esprimibile come differenza di funzioni nonnegative, nondecrescenti e limitate (ovvero, $\exists \psi^{*} // c(\psi) > 0\, \forall\, \psi \in \!R \land ( \partial \theta/\partial\psi\,\geq0 \left[-\infty,\psi^{*}\right] \lor \partial \theta/\partial\psi<0\, \left[\psi^{*},+\infty\right]) $), che:
		\begin{description}\itemsep0pt
			\item [capacità specifica (jordan)]: $c(\psi)=p(\psi) - q(\psi)$
			\item [per $\psi\leq\psi^{*}$] ho $p(\psi)=c(\psi)$ e $q(\psi)=0$
			\item [per $\psi>\psi^{*}$] ho $p(\psi)=c(\psi^{*})$ e $q(\psi)=p(\psi)-c(\psi)$
			\item [volumi d'acqua risultanti]: $\theta(\psi) = \theta_{1}(\psi) - \theta_{2}(\psi)$, con\newline
				\begin{equation}
					\theta_{1} = \theta_{r} + \int_{-\infty}^{\psi}p(\eta)d\eta
					\label{eq:theta1expr}
				\end{equation}
				e 
				\begin{equation}
					\theta_{2} = \int_{-\infty}^{\psi}q(\eta)d\eta
					\label{eq:theta2expr}
				\end{equation}
		\end{description}
		Quindi, i volumi fluidi si possono esprimere come:
		\begin{equation}
			\begin{align}
				\theta_{1}(\psi) = \theta(\psi), & \theta_{2}=0 && \psi\leq\psi^{*} \\ 
				\theta_{1}(\psi) = \theta(\psi^{*}) + c(\psi^{*})(\psi-\psi^{*}), & \theta_{2}=\theta_{1}-\theta(\psi) && \psi>\psi^{*}
			\end{align}
		\end{equation}
		\begin{figure}[!h]
		\centering
		\begin{tikzpicture}
		  	\begin{axis}[ 
				  no markers, domain=0:8, samples=100,
				  axis lines*=left, xlabel=$\psi$, ylabel=$\textcolor{cyan!60!black}{c(\psi)} / \textcolor{red!60!black}{\theta(\psi)}$,
				  every axis y label/.style={at=(current axis.above origin),anchor=south},
				  every axis x label/.style={at=(current axis.right of origin),anchor=west},
				  height=5cm, width=8cm,
				  xtick={4},
    			  xticklabels={$\psi^{*}$},				  
    			  ytick={0.4},
    			  yticklabels={$\frac{\partial c}{\partial\psi}=0$},				      			  
				  enlargelimits=true, clip=false, axis on top
		  	] 
			\addplot [very thick,cyan!60!black] {gauss(4,1)};
			\addplot [very thick,red!60!black] {cdf(4,1)};
			\end{axis}
		\end{tikzpicture}
		\caption{Esempio di andamento (gaussiano) della \textcolor{cyan!60!black}{capacità specifica} decomponibile secondo Jordan. Si noti la positività della stessa
		$\forall\, \mathbb{R} $, la non decrescenza di $\frac{\partial \theta}{\partial \psi}$ per $\psi\leq\psi^{*}$ e la decrescenza di $\frac{\partial \theta}{\partial \psi}$ per $\psi>\psi^{*}$. Per riferimento è plottata anche la \textcolor{red!60!black}{distribuzione cumulata della capacità specifica} (una curva sigmoide: notare la forse interessante similitudine con la forma delle curve di van Genuchten \cite{Genuchten1980}).}
		\end{figure}

		%\lstinputlisting[language=Fortran, firstline=71, lastline=80]{../BASE_CODE/Rheology.f90}
	
		\subroutines
			%\lstinputlisting[language=Fortran, firstline=76, lastline=76]{../BASE_CODE/Rheology.f90} La \emph{subroutine}
			 calcola la $p$ ($\frac{\partial\theta_{1}}{\partial\psi}=p$) della decomposizione di Jordan in caso di $\psi\leq\psi^{*}$, passando $\psi$;
			%\lstinputlisting[language=Fortran, firstline=78, lastline=78]{../BASE_CODE/Rheology.f90} La \emph{subroutine}
			 calcola la $p$ della decomposizione di Jordan in caso di $\psi>\psi^{*}$, passando $\psi^{*}$;			 
	%------------------------------------------------------------------------------------
	% SUBROUTINE DTHETA2
	%------------------------------------------------------------------------------------
	\subsection{dTheta2}
		La funzione calcola la derivata del secondo elemento della decomposizione di Jordan $q(\psi) = p(\psi) - c(\psi)$ della capacità specifica ($q=\frac{\partial \theta_{2}}{\partial \psi}$).
		Si veda \ref{subs:theta1} per ulteriori dettagli.
		%\lstinputlisting[language=Fortran, firstline=82, lastline=90]{../BASE_CODE/Rheology.f90} 
		\subroutines
			%\lstinputlisting[language=Fortran, firstline=86, lastline=86]{../BASE_CODE/Rheology.f90} La \emph{subroutine} provvede al calcolo
			della $c(\psi)$
			\newline
			%\lstinputlisting[language=Fortran, firstline=87, lastline=87]{../BASE_CODE/Rheology.f90} La \emph{subroutine} provvede al calcolo
			della $p(\psi)$
	%------------------------------------------------------------------------------------
	% SUBROUTINE THETA1
	%------------------------------------------------------------------------------------
	\subsection{Theta1}
	\label{sub:theta1}
		La funzione calcola la frazione di contenuto d'acqua del volume di controllo associata all'integrale 
		del primo elemento della decomposizione di Jordan per la capacità specifica. Per $\psi\leq\psi^{*}$, la funzione
		procede a chiamare la procedura per il calcolo di $\theta(\psi)_{1}$ derivata dall'applicazione diretta della
		parametrizzazione di van Genuchten \ref{sub:thetaf}, mentre oltre $\psi_{*}$ (punto in cui $\frac{\partial c(\psi)}{\partial \psi}$ è massima) somma a $\theta_1{\psi<\psi_{*}}$ un integrale alle differenze (secondo Riemann) della capacità specifica 
		(\emhp{e.g.}, $\theta\approx\Delta\psi*\Delta c(\psi)$).  
		%\lstinputlisting[language=Fortran, firstline=34, lastline=47]{../BASE_CODE/Rheology.f90}
		\subroutines
			%\lstinputlisting[language=Fortran, firstline=43, lastline=43]{../BASE_CODE/Rheology.f90} La \emph{subroutine}
			calcola il contenuto d'acqua secondo van Genuchten;
			%\lstinputlisting[language=Fortran, firstline=44, lastline=44]{../BASE_CODE/Rheology.f90} La \emph{subroutine}
			calcola la capacità specifica $q(\psi)$.

	%------------------------------------------------------------------------------------
	% SUBROUTINE THETA2
	%------------------------------------------------------------------------------------
	\subsection{Theta2}
		La funzione calcola la frazione di contenuto d'acqua del volume di controllo associata all'integrale 
		del secondo elemento della decomposizione di Jordan per la capacità specifica, come
		differenza tra il contenuto d'acqua totale e $\int_{-\infty}^{\psi}p(\eta)d\eta = \theta_{1}$.
		%\lstinputlisting[language=Fortran, firstline=49, lastline=57]{../BASE_CODE/Rheology.f90}
		\subroutines
			%\lstinputlisting[language=Fortran, firstline=53, lastline=53]{../BASE_CODE/Rheology.f90} La \emph{subroutine}
			calcola il contenuto d'acqua totale tramite la parametrizzazione di van Genuchten.
			%\lstinputlisting[language=Fortran, firstline=54, lastline=54]{../BASE_CODE/Rheology.f90} La \emph{subroutine}
			calcola l'integrale di $p(\psi)$.

	%------------------------------------------------------------------------------------
	% SUBROUTINE V
	%------------------------------------------------------------------------------------
	\subsection{V}
		%\lstinputlisting[language=Fortran, firstline=92, lastline=100]{../BASE_CODE/Rheology.f90}

	%------------------------------------------------------------------------------------
	% SUBROUTINE DV
	%------------------------------------------------------------------------------------
	\subsection{dV}
		%\lstinputlisting[language=Fortran, firstline=102, lastline=110]{../BASE_CODE/Rheology.f90}

\section{MODULE Gradiente coniugato - CGModule}
	Per una introduzione al metodo del Gradiente Coniugato, si veda \cite{Shewchuk1994}.

\section{Spunti di riflessione}
\begin{itemize}\itemsep0pt
	\item A rigore, l'equazione di Richards dipende dalla viscosità $\mu$ del fluido: in \cite{Casulli2010} sembra si supponga che il fluido sia sempre acqua a temperatura ambiente, e la sua viscosità è inglobata all'interno della K\dots Volendo, forse si potrebbe accoppiare a Richards il bilancio dell'energia considerando che $\mu=\mu(T)$\dots
\end{itemize}

%----------------------------------------------------------------------------------------
%	BIBLIOGRAPHY
%----------------------------------------------------------------------------------------

\renewcommand{\refname}{\spacedlowsmallcaps{Riferimenti}} % For modifying the bibliography heading

\bibliographystyle{unsrt}
\bibliography{sample.bib} % The file containing the bibliography

%----------------------------------------------------------------------------------------


\end{document}